\documentclass{article}
\usepackage[utf8]{inputenc}

\title{Numerical Methods}
\author{Raghvendra Mishra}

\date{4th Semester}

\begin{document}
\maketitle

\section{Computer Arithmetic}

\subsection{Significant Numbers}

Digits used to express a number. Such that 0.0036, 0.000587, 0.000000296 all have
only 3 significant figures since the 0's only help to fix the position 
of the decimal point.

\[3.9\times10^6\to 2\textup{ significant digits}\]
\[3.909\times10^6\to 4\textup{ significant digits}\]
\[3\times10^6\to 1\textup{ significant digits}\]


\subsection{Floating Point Numbers}

Any integer $ \beta > 1 $ can be used as the base for a number system. 
Now, since the typcial computer works in binary, there's certain errors 
while rounding off.
For example a simple number as $ \frac{1}{10} $ requires an infinite binary
expression: \[\frac{1}{10} = (0.00011..)_2\]
But in computer it will return 
\[0.10000 00014 90116 11938 47656 25000 00000 00000\]
if you print 0.1 from a 32 bit workstation



\end{document}